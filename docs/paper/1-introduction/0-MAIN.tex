\section{Einleitung}
\label{sec:introduction}

Das Forschungsprojekt \glqq{}NuData Campus\grqq{}, welches an der Hochschule München für angewandte Wissenschaften durchgeführt wird, hat die Nutzungsdaten-basierte Optimierung von Gebäuden und Anlagen, am Beispiel der Hochschule selbst, zum Ziel~\cite{NuDataCampus}.
Im Zuge laufender Untersuchungen wird unter anderem ein Clustering, auf Basis von erhaltenen Daten aus Data-Mining Verfahren, für verschiedene Räume in Gebäuden durchgeführt.
Die betrachteten Räume können so zu Gruppen von Räumen mit ähnlichen Eigenschaften zugeordnet werden.

\subsection{Ziel der Arbeit}
\label{subsec:purpose}

Die gewonnenen Clustering-Ergebnisse sind je nach Größe des betrachteten Gebäudes unübersichtlich.
Um sich einen Überblick über die Daten zu verschaffen, ist ein Werkzeug zur Visualisierung der Gebäudepläne sowie des dazugehörigen Clustering-Ergebnisse nötig.
In dieser Arbeit sollen dementsprechend die Anforderungen gesammelt und ein passendes Anwendungssystem entwickelt werden.

\subsection{Hintergrund}
\label{subsec:background}

Das Projekt wird im Rahmen der Veranstaltung \textit{\glqq{}Aktuelle Forschungsprojekte in der Wirtschaftsinformatik\grqq{}} bei Prof. Dr. Peter Mandl im Wintersemester \textit{20/21} an der Hochschule München bearbeitet.
Dabei betreut Herr Manuel Weber (\textit{manuel.weber@hm.edu}) die Arbeit während des Entwicklungsprozesses.

\subsection{Aufbau der Arbeit}
\label{subsec:structure}

Einleitend wollen wir bereits vorhandene CAD-Viewer Programme betrachten und vergleichen, mit dem Ziel übliche Funktionalitäten in dieser Anwendungskategorie zu finden.
Basierend auf den gewonnenen Erkenntnissen soll eine Anforderungssammlung durchgeführt werden.
Nachfolgend argumentieren wir die zu verwendenden Werkzeuge und Technologien, anhand der funktionalen sowie nicht-funktionalen Anforderungen.
Anschließend präsentieren wir die Ergebnisse der Arbeit, also die finale Architektur und eine Dokumentation der Anwendung aus Benutzersicht.
Abschließend fassen wir die Eindrücke des Projekts zusammen und geben Vorschläge für weitere Entwicklungen.
