\subsection{Anforderungssammlung}
\label{subsec:requirements}

Nun wollen wir die Anforderungen für die neue Anwendung gesammelt auflisten.
Dabei listen wir sowohl \textit{funktionale}, als auch \textit{nicht-funktionale}, Anforderungen auf.

\vspace{8pt}

\subsubsection{Funktionale Anforderungen}
\label{subsubsec:functional-requirements}

\begin{center}
    \small
    \begin{supertabular}{ p{1cm}|p{3cm}|p{4cm} }
        Nr.             & Anforderung                                   & Weitere Informationen                                                                                                                                                                            \\
        \hline
        \textbf{A1}     & Anzeigen einer 2-dimensionalen *.dxf Datei    & Die Anwendung soll Dateien im DXF Format laden und darstellen können.                                                                                                                            \\
        \textbf{A2}     & Navigation in der Benutzeroberfläche          & Gerade größere Gebäudepläne können unübersichtlich sein, daher muss eine geeignete Navigation möglich sein.                                                                                      \\
        \textbf{A2.1}   & Verschieben der Arbeitsfläche                 & Die Arbeitsfläche soll per Mausklick (linke Maustaste) und Ziehen verschoben werden können.                                                                                                      \\
        \textbf{A2.2}   & Verkleinern und Vergrößern der Arbeitsfläche  & Die Arbeitsfläche soll mit dem Mausrad vergrößert und verkleinert werden.                                                                                                                        \\
        \textbf{A2.2.1} & Verkleinern der Arbeitsfläche                 & Bewegung des Mausrads nach unten soll die Arbeitsfläche (den Abbildungsmaßstab) verkleinern.                                                                                                     \\
        \textbf{A2.2.2} & Vergrößern der Arbeitsfläche                  & Bewegung des Mausrads nach oben soll die Arbeitsfläche (den Abbildungsmaßstab) vergrößern.                                                                                                       \\
        \textbf{A3}     & Anzeige eines Clusteringergebnisses für Räume & Als Differenzierung zu anderen CAD-Viewer Programmen sollen bestimmte Räume auf Basis eines Clustering Ergebnisses gesondert hervorgehoben werden.                                               \\
        \textbf{A4}     & Export des Gebäudeplans                       & Der Gebäudeplan und die Markierungen/Hervorhebungen durch das eingelesenen Clusteringergebnis sollen in eine Datei exportiert werden können, welche zum einfachen Teilen verschickt werden kann. \\
        \textbf{A5}     & REST-API zum Hochladen von Raummappings (Clusteringergebnis) und CAD-Dateien & Zur vereinfachten Bereitstellung der Clusteringergebnisse (Raummappings) und CAD Dateien im Viewer Programm, sollen die Dateien über eine REST API hochgeladen werden können. \\
    \end{supertabular}
\end{center}

\vspace{10pt}

\subsubsection{Nichtfunktionale Anforderungen}
\label{subsubsec:non-functional-requirements}

\begin{center}
    \small
    \begin{supertabular}{ p{1cm}|p{3cm}|p{4cm} }
        Nr. & Anforderung & Weitere Informationen \\
        \hline
        \textbf{NA1} & Einfache und schnelle Bereitstellung/Installierbarkeit & Die Anwendung soll schnell und einfach bereitgestellt bzw. installiert werden können. Als Vergleichsmerkmale hierzu sollen die Installationszeit und Dateigrößen der untersuchte Anwendung \textit{DWG TrueView} auf demselben Rechner dienen. \\
        \textbf{NA2} & Einfache Erweiterbarkeit und Wartbarkeit & Während der Implementierung soll vor allem auf die Möglichkeit einer späteren einfachen Erweiterung der Funktionalität geachtet werden. Außerdem soll der Quellcode durch erklärende Kommentare besser wartbar werden. \\
        \textbf{NA3} & Übersichtliche Darstellung & Die Benutzeroberfläche wie auch die Visualisierung des Gebäudeplans sollen möglichst übersichtlich und intuitiv zu bedienen sein. \\
        \textbf{NA4} & Ruckelfreie Navigation im Gebäudeplan & Die Arbeitsfläche/Der Gebäudeplan soll ruckelfrei über die Anwendung navigiert werden können. Konkret bedeutet das min. 30 Bilder pro Sekunde auf einem Referenzgerät (Hier nicht näher spezifiziert). \\
    \end{supertabular}
\end{center}
