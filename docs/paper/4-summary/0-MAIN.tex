\section{Zusammenfassung und Fazit}
\label{sec:summary}

Wir haben im Laufe des Projekts eine Webanwendung erstellt, mit welcher die im Forschungsprojekt \glqq{}NuData Campus\grqq{} erhaltenen Clustering-Ergebnisse geeignet visualisiert werden können.
Die exportierte Webanwendung mitsamt CAD-Datei und Raumzuordnung kann außerdem per HTML Datei mit beteiligte Personen geteilt werden.

Nachdem im vorherigen Kapitel die Umsetzung der funktionalen Anforderungen betrachtet wurde, wollen wir zum Abschluss noch einmal die Erfüllung der nicht-funktionalen Anforderungen überprüfen.
Während der Entwicklung ist besonders auf die Erweiterbarkeit und Wartbarkeit (\textbf{NA2}) der Anwendung geachtet worden, sodass auch in Zukunft nötige Änderungen vorgenommen werden können.
Wie bereits erläutert ist die Erweiterung um einen neuen CAD-Dateityp bereits architektonisch vorgesehen.
Des Weiteren sind jegliche Methoden und Variablen im Quellcode der Anwendung dokumentiert.
Aus unserer Sicht ist die Anwendung auch übersichtlich und intuitiv zu bedienen (\textbf{NA3}).
So schlägt die Startansicht bereits die üblichen nächsten Schritte vor.
Des Weiteren beschreiben Tooltips alle möglichen Knöpfe in der Anwendung.
Die Dialoge sind als Wizards konzipiert und leiten den Benutzer durch die einzelnen Schritte, wie beispielsweise beim Hochladen einer CAD-Datei.
Ebenfalls sind auf unserem Entwicklungssystem - einem Notebook mit aktuellem 6-Kern Prozessor und integrierter Grafikeinheit - die bereitgestellten Gebäudepläne flüssig navigierbar, wie von Anforderung \textbf{NA4} verlangt.
Selbstverständlich ist die Anwendung, da eine Webanwendung, schneller dem Benutzer bereitzustellen, als das mehrere Gigabyte schwere DWG TrueView.
Es reicht die Eingabe einer URL in der Adresszeile des Browsers, wodurch auch Anforderung \textbf{NA1} erfüllt ist.

Abschließend geben wir noch eine Idee zur zukünftigen Erweiterung der entstandenen Webanwendung mit.
So ist die Erweiterung der verwendeten DXF-Parser Bibliothek um das Parsen der sogenannten \texttt{ATTRIB} Entitäten sinnvoll.
Die Erweiterung ermögliche die Raumzuordnung anhand der in der DXF-Datei festgesetzten Annotationspositionen pro Raum.
Diese Annotationen sind in den meisten Fällen innerhalb der Räume und erlauben zumindest in diesen Fällen eine Zuordnung.
Ein weiterer Vorteil der \texttt{ATTRIB} Entitäten ist, dass wir die Raumnamen an die richtige Position auf dem Gebäudeplan zeichnen können.
Bisher werden die Raumnamen lediglich in die Mittel der einzelnen Vertices eines Raumes gezeichnet, was teilweise zu unschönen Ergebnissen führt.
