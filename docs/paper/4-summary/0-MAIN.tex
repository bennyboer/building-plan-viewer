\section{Zusammenfassung und Fazit}
\label{sec:summary}

Wir haben im Laufe des Projekts eine Webanwendung erstellt, mit welcher die im Forschungsprojekt \glqq{}NuData Campus\grqq{} erhaltenen Clustering-Ergebnisse geeignet visualisiert werden können.
Die exportierte Webanwendung mitsamt CAD-Datei und Raumzuordnung kann außerdem per HTML Datei mit beteiligte Personen geteilt werden.

Während der Entwicklung ist besonders auf die Erweiterbarkeit der Anwendung geachtet worden, sodass auch in Zukunft nötige Änderungen vorgenommen werden können.
Eine konkrete Verbesserung wäre die Erweiterung der verwendeten DXF-Parser Bibliothek um das Parsen der sogenannten \texttt{ATTRIB} Entitäten.
Die Erweiterung ermögliche die Raumzuordnung anhand der in der DXF-Datei festgesetzten Annotationspositionen pro Raum.
Diese Annotationen sind in den meisten Fällen innerhalb der Räume und erlauben zumindest in diesen Fällen eine Zuordnung.
Ein weiterer Vorteil der \texttt{ATTRIB} Entitäten ist, dass wir die Raumnamen an die richtige Position auf dem Gebäudeplan zeichnen können.
Bisher werden die Raumnamen lediglich in die Mittel der einzelnen Vertices eines Raumes gezeichnet, was teilweise zu unschönen Ergebnissen führt.
